\documentclass[11pt,a4paper,twoside,twocolumn]{article}
\usepackage[T1]{fontenc}
\usepackage{amsmath,amssymb,amsthm,mathtools}
\usepackage{siunitx}
\usepackage{geometry}
  \geometry{
  a4paper,
  left = 0.5in,
  right = 0.5in,
  top = 0.75in,
  bottom = 0.75in}
\setlength\columnsep{0.5in}
\usepackage{parskip}
  \renewcommand{\baselinestretch}{1}
  \setlength{\parindent}{1em}
\usepackage{soul}
\usepackage{authblk}
\renewcommand*{\Authfont}{\small}
\renewcommand*{\Affilfont}{\small}
\renewcommand*{\Authsep}{, }
\renewcommand*{\Authand}{, }
\renewcommand*{\Authands}{, }
\usepackage{graphicx}
\usepackage[font=it,labelfont=it]{caption}
\usepackage{tabularx}
\usepackage{multirow}
\usepackage{multicol}
\usepackage{makecell}
\usepackage[
  backend=biber,
  style=nature,
  url=false,
  eprint=false,
  doi=true,
  giveninits,
  defernumbers=true
]{biblatex}
  \urlstyle{same}
  \renewcommand*{\bibfont}{\footnotesize}
\usepackage[libertine]{newtxmath}
\usepackage{titlesec}
\usepackage[no-math]{fontspec}
\setmainfont[
  Mapping=tex-text,
  Path=./fonts/,
  BoldFont       = CalibriB.ttf,
  ItalicFont     = CalibriI.ttf,
  BoldItalicFont = CalibriBI.ttf]{Calibri.ttf}
\setsansfont[
  Path=./fonts/,
  BoldFont       = CalibriB.ttf,
  ItalicFont     = CalibriI.ttf,
  BoldItalicFont = CalibriBI.ttf]{Calibri.ttf}
\newfontfamily\calibri[
  Path=./fonts/,
  BoldFont       = CalibriB.ttf,
  ItalicFont     = CalibriI.ttf,
  BoldItalicFont = CalibriBI.ttf]{Calibri.ttf}
\defaultfontfeatures{Mapping=tex-text}
\titleformat{\section}{\bfseries\MakeUppercase}{\thesection.}{1em}{}
\titleformat*{\subsection}{\itshape}
\titleformat{\subsubsection}{}{\ul{\thesubsubsection\enspace}}{-0.1em}{\ul}
\titleformat*{\paragraph}{\bfseries\itshape}
\titlespacing*{\section}{0pt}{1em}{0.5em}
\titlespacing*{\subsection}{0pt}{0.5em}{0.5em}
\titlespacing*{\subsubsection}{0pt}{0.5em}{0.5em}
\usepackage{fancyhdr}
  \pagestyle{fancy}
  \fancyhf{}
  \renewcommand{\headrulewidth}{0pt}
  \chead{\calibri\textbf{\Large Energy Proceedings} \\2022}
  \rhead{\calibri\footnotesize ISSN 2004-2965 \vspace{0.5em}}
  \cfoot{\calibri\small\thepage}
%%
\title{\bfseries Template for Preparing a Paper for ICAE2022/Energy Proceedings (APA style)\footnote{This is a paper for the 14th International Conference on Applied Energy - ICAE2022, 8-11 Aug 2022, Bochum, Germany.}}
\author[1]{Author A}
\author[1,2,*]{Author B}
\affil[1]{Affiliation of author A}
\affil[2]{Affiliation of author B}
\affil[*]{Corresponding author.}
\date{}
%
\begin{document}
\maketitle
\thispagestyle{fancy}

\begin{abstract}
    This is the template of an ICAE2022 conference paper to provide author’s instruction on how to prepare a manuscript. The paper will be considered for the publication in Energy Proceedings. An abstract normally consists of about 100 words, giving a brief account of the most relevant aspects of the paper.
\end{abstract}

\section*{Nomenclature}
\begin{tabularx}{\linewidth}{|cX|}
\hline
    \multicolumn{2}{|l|}{\itshape Abbreviations}\\
    EP & Energy Proceedings \\
    \multicolumn{2}{|l|}{\itshape Symbols}\\
    $n$ & Year  \\
\hline
\end{tabularx}

\section{Introduction}
We would like to warmly invite you to prepare a paper in ICAE2022/Energy Proceedings. Energy Proceedings is a peer-reviewed, open-access high-quality serial publication with bimonthly/quarterly releases, currently 5-6 volumes annually. Energy Proceedings covers a broad field of multidisciplinary subjects in energy sciences and 
technologies. This includes energy-related economics and social sciences, as well as policy and legal studies.
Energy Proceedings is to publish papers, which are new and exciting in energy research and development that offer opportunities for translation into sustainable solutions.
All papers published in ICAE2022/Energy Proceedings will be reviewed organized by editorial board and editors...

\section{Paper structure}
\subsection{Subdivision - numbered sections}
A paper submitted to ICAE2022/Energy Proceedings should not exceed ten pages. Please use this template to prepare your paper. Font Calibri should be used with the size of 11. Figures and tables should be embedded and not supplied separately.
Divide your article into clearly defined and numbered sections. Subsections should be numbered 1.1 (then 1.1.1, 1.1.2, ...), 1.2, etc. (the abstract is not included in section numbering). Use this numbering also for internal cross-referencing: do not just refer to 'the text'. Any subsection may be given a brief heading. Each heading should appear on its own separate line.

\subsection{Introduction}
Introduction presents background information on the objectives, research questions and scope/limitation of the paper. 

...

...

\subsection{Material and methods}
Provide sufficient detail to allow the work to be reproduced. Methods already published should be indicated by a reference: only relevant modifications should be described.

\subsection{Theory/calculation}
A Theory section should extend, not repeat, the background to the article already dealt with in the Introduction and lay the foundation for further work. In contrast, a Calculation section represents a practical development from a theoretical basis.

\subsection{Results}
Results should be clear and concise.

\subsection{Discussion}
This should explore the significance of the results of the work, not repeat them. A combined Results and Discussion section is often appropriate. Avoid extensive citations and discussion of published literature.

\subsection{Conclusions}
The main conclusions of the study may be presented in a short Conclusions section, which may stand alone or form a subsection of a Discussion or Results and Discussion section.

\subsection{References} 
\subsubsection{Citation in text}
Any references cited in the abstract must be given in full. Unpublished results and personal communications are not recommended in the reference list, but may be mentioned in the text. If these references are included in the reference list, they should follow the standard reference style of the journal and should include a substitution of the publication date with either 'Unpublished results' or 'Personal communication' Citation of a reference as 'in press' implies that the item has been accepted for publication.

\subsubsection{Web references}
As a minimum, the full URL should be given and the date when the reference was last accessed. Any further information, if known (DOI, author names, dates, reference to a source publication, etc.), should also be given. Web references can be listed separately (e.g., after the reference list) under a different heading if desired. 

\section{Conclusions}

\section*{Acknowledgement}
This unofficial template is created by T.Y. Cho and available on Github. It has been tested to work with XeLaTex and LuaLaTex, and feel free to use it. Please make sure you are legally licenced to use Calibri font, the creator is not responsible for any legal liability resulted from using this template.

\printbibliography
\end{document}